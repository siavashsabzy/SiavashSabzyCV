%%%%%%%%%%%%%%%%%%%%%%%%%%%%%%%%%%%%%%%%%
% "ModernCV" CV and Cover Letter
% LaTeX Template
% Version 1.11 (19/6/14)
%
% This template has been downloaded from:
% http://www.LaTeXTemplates.com
%
% Original author:
% Xavier Danaux (xdanaux@gmail.com)
%
% License:
% CC BY-NC-SA 3.0 (http://creativecommons.org/licenses/by-nc-sa/3.0/)
%
% Important note:
% This template requires the moderncv.cls and .sty files to be in the same 
% directory as this .tex file. These files provide the resume style and themes 
% used for structuring the document.
%
%%%%%%%%%%%%%%%%%%%%%%%%%%%%%%%%%%%%%%%%%

%----------------------------------------------------------------------------------------
%	PACKAGES AND OTHER DOCUMENT CONFIGURATIONS
%----------------------------------------------------------------------------------------

\documentclass[10pt,a4paper,sans]{moderncv} % Font sizes: 10, 11, or 12; paper sizes: a4paper, letterpaper, a5paper, legalpaper, executivepaper or landscape; font families: sans or roman

\moderncvstyle{banking} % CV theme - options include: 'casual' (default), 'classic', 'oldstyle' and 'banking'
\moderncvcolor{blue} % CV color - options include: 'blue' (default), 'orange', 'green', 'red', 'purple', 'grey' and 'black'

\usepackage{lipsum} % Used for inserting dummy 'Lorem ipsum' text into the template
\usepackage{multicol}
\usepackage{graphicx}
\usepackage[scale=0.9]{geometry}%\usepackage{cmcyr}
 % Reduce document margins
%\setlength{\hintscolumnwidth}{3cm} % Uncomment to change the width of the dates column
%\setlength{\makecvtitlenamewidth}{10cm} % For the 'classic' style, uncomment to adjust the width of the space allocated to your name
\usepackage{fontawesome}
%----------------------------------------------------------------------------------------
%	NAME AND CONTACT INFORMATION SECTION
%----------------------------------------------------------------------------------------

\firstname{Siavash} % Your first name
\familyname{Sabzy} % Your last name

% All information in this block is optional, comment out any lines you don't need
\title{Curriculum Vitae}
\address{}{\textbf{Iran University of Science and Technology -- School of New Technologies}}




\mobile{(+98) 912 082 4919}
%\phone{(000) 111 1112}
%\fax{(000) 111 1113}
\email{ siavash\_sabzy@alumni.iust.ac.ir }
%\homepage{www.alihejazi.wordpress.com} % The first argument is the url for the clickable link, the second argument is the url displayed in the template - this allows special characters to be displayed such as the tilde in this example
\extrainfo{Date of Birth: Sep, 14, 1993: 28 years}
%\social[linkedin][www.linkedin.com/in/abolfazl-taghribi-b532%1680?trk=hp-identity-name]{Abolfazl Taghribi}
%\photo[70pt][0.2pt]{pictures/picture} % The first bracket is the picture height, the second is the thickness of the frame around the picture (0pt for no frame)
%\quote{"A witty and playful quotation" - John Smith}

%----------------------------------------------------------------------------------
%            content
%----------------------------------------------------------------------------------
\makeatletter
\newcommand*{\rom}[1]{\expandafter\@slowromancap\romannumeral #1@}
\makeatother
\begin{document}
\makecvtitle
\vspace{-35 pt}
%\centerline{\includegraphics[width=4.5cm, height=6cm]{1}}
%----------------------------------------------------------------------------------------
%	EDUCATION SECTION
%----------------------------------------------------------------------------------------
%\section{Objective}

%Pursuing graduate studies in Power with an emphasis on Power Electronics and Power System towards a PhD degree and beyond so as to acquire sufficient knowledge and experience for a productive life time career in teaching and applied research.

%----------------------------------------------------------------------------------------
%	RESEARCH INTERESTS SECTION
%----------------------------------------------------------------------------------------


\section{Research Interests}



\begin{tabular}{ll}

%\item{Power Quality}
%\item{Image and video processing}

Astrodynamics  \hspace{12em} & Machine Learning \\
Orbit/Attitude Dynamics and Control \hspace{12em} & Orbit/Attitude Determination  \\
Numerical Calculations \hspace{12em}   \\


\end{tabular}



\section{Education}

\begin{itemize}
	\item \textbf{Master of Science} \hfill \textbf{Satellite Technology Engineering} \\
	\href{http://www.iust.ac.ir/en}{\textbf{Iran University of Science and Technology, Tehran, IR}} \hfill \textbf{Sep. 2017 - Jan. 2020}\\ GPA: 3.42/4 (17.10 / 20) \\ Thesis: "Coupled Orbit and Attitude Dynamics of a Spacecraft in the Ecliptic Restricted Three Body Problem" \\ Supervisor: Dr. Kamran Daneshjoo
	\par 
	Advisor: Dr. Majid Bakhtiari
	
%	\item \textbf{Master of Science} \hfill \textbf{2015--2016} \\
%	\href{http://www.sharif.ir/web/en}{\emph{Sharif University of Technology}} \hfill \emph{Tehran-Iran}
%	\begin{itemize}
%		\item \textbf{Major: Power Systems} \hspace{10 pt} 
%		\item I didn't complete the program and decided to attend University of Alberta.
%		\\
%	\end{itemize}
	
	\item \textbf{Bachelor of Science} \hfill \textbf{Mechanical Engineering} \\
	\href{https://sru.ac.ir/en}{\textbf{Shahid Rajaee University, Tehran, IR}} \hfill \textbf{Jan. 2013 - Jan. 2017} \\  Thesis: "Vibration Analysis of a Rotary Shaft with Rigid or Flexible Bearings by Considering the Rotor Gyroscopic Effects" \\ Supervisor: Dr. Majid Shahgholi
	
	
	\item \textbf{High School} \hfill \textbf{Mathematics and physics } \\
	{\textbf{Alameh Tabatabaei High School}} \hfill \textbf{Sep. 2007 - June. 2010} \\ Aleshtar, Lorestan, Iran   %GPA: 4/4 (18.51/ 20) 

	%\begin{itemize}
	%	\item \textbf{Subfield: Power} \hspace{51 pt} ~~~~Total Average: 15.71/20 via 141 credits
	%\end{itemize}
	
	%\item \textbf{High School} \hfill \textbf{2007--2011} \\
	%\emph{Shahid Beheshti} \hfill \emph{Bushehr-Iran}\\
	%Affiliated with the National Organization for the Development of Exceptional Talents (NODET)

\end{itemize}

\section{Some Courses}
\begin{multicols}{2}  % makes the text in two columns
	\begin{itemize}
		
		\item Celestial Mechanics and Orbital Dynamics   	 	\hfill{18.25/20}
		\item Satellite Attitude Determination  					 	\hfill{20/20}              
		\item Special Courses in Satellite Technology (Sensors)             		 	\hfill{19.5/20}
		\item Satellite Attitude Control            	 	\hfill{19.5/20} 
		
	\end{itemize}
\end{multicols}

\section{Online Courses}

 % makes the text in two columns
	\begin{itemize}
		\item Machine Learning (Stanford University)  \textbf{\href{https://coursera.org/verify/BCTCN2JXRGRD} {\underline{Coursera certification}}}	\hfill{98.82/100}
		\item Fundamentals of Reinforcement Learning (University of Alberta)  \textbf{\href{https://coursera.org/verify/9BYJ37J7UPAY} {\underline{Coursera certification}}}  					 	\hfill{98.75/100} 
		\item Kinetics: Studying Spacecraft Motion (University of Colorado Boulder)  \textbf{\href{https://coursera.org/verify/E9RVRNYBTUFB} {\underline{Coursera certification}}}	\hfill{100/100}
		\item Kinematics: Describing the Motions of Spacecraft (University of Colorado Boulder)  \textbf{\href{https://coursera.org/verify/G53HFNRJ9JBL} {\underline{Coursera certification}}}	\hfill{93.26/100}
		\item Spacecraft Dynamics Capstone: Mars Mission (University of Colorado Boulder)  \textbf{\href{https://coursera.org/verif y/DDJTUCGB66BG} {\underline{Coursera certification}}}	\hfill{96.33/100}
		\item Control of Nonlinear Spacecraft Attitude Motion (University of Colorado Boulder)  \textbf{\href{....} {\underline{Coursera certification}}}	\hfill{Ongoing}
		\item Sample-based Learning Methods (University of Alberta)  \textbf{\href{...} {\underline{Coursera certification}}}  					 	\hfill{Ongoing} 		
	     
	\end{itemize}


\section{Publications(\href{https://www.researchgate.net/profile/Siavash_Sabzy/publications}{"\underline{Click to see}")}}
\begin{itemize}

\item\textbf{Journals:}
\item Siavash Sabzy, Kamran Daneshjou, Majid Bakhtiari 
 \textbf{\href{https://doi.org/10.1016/j.asr.2021.01.019}{"\underline{Periodic attitude motions along planar orbits in the elliptic restricted} \\ \underline{three-body problem}"}}, Advances in Space Research, Elsevier. \textbf{(Published)}
\item Siavash Sabzy, Majid Bakhtiari, Elyas Rashno 
 \textbf{\href{......}{"\underline{Distinguishing Periodic Attitude Motions from Poincaré Sections Using } \\ \underline{a Compatible Clustering Method}"}}, Submitted to Acta Astronautica.\\

\item\textbf{Conferences:}
\item Siavash Sabzy, Bahman Ghorbani Vaghei
 \textbf{\href{https://civilica.com/doc/881918/}{"\underline{Designing Coupled Attitude and Orbit Control System of GEO Satellite During} \\ \underline{Orbit Transfer"}}}, 2018 (DMECONF04). \textbf{(Published)(in Persian)}
 
 \item Siavash Sabzy, Majid Bakhtiari, Kamran Daneshjou
 \textbf{"\underline{Investigating the Effect of Eccentricity and Mass Ratio of Primaries} \\ \underline{ on the Structure of Lyapunov Orbits"}}, The 19th International Conference of Iranian Aerospace Society. \textbf{(Accepted)}
 
 \item Siavash Sabzy, Meisam Farajollahi
 \textbf{"\underline{Dynamical Simulation of MEMS Inertial Sensor for Measuring the Gravity Gradient } \\ \underline{ Torque in Low Earth Orbit"}}, The 19th International Conference of Iranian Aerospace Society. \textbf{(Accepted)(in Persian)}



%\item\textbf{"A Novel Fast Bio-Inspired Feature for Motion Estimation"}, 10th Iranian Conference on Machine Vision \& image processing, Abolfazl Taghribi, Abolghasem A. Raie, Majid Shalchian,DOI:10.1109/IranianMVIP.2017.8342366, \textbf{\href{https://ieeexplore.ieee.org/abstract/document/8342366/}{(Link to the paper)}}(accepted for oral presentation)
\end{itemize}


%\item \textbf{Permitted to study Communication as a minor} (This permission is only awarded to talented students, introduced by the Exceptional Talents Office).\vspace{1mm}
%\item Granted admission from \textbf{Talented Student Office} of Amirkabir University of Technology for graduate study.\vspace{1mm}
%\item  Semi-finalist in \textbf{National Mathematics Olympiad} in two successive years at high school.\vspace{1mm}
%\item  Semi-finalist in \textbf{National Physics Olympiad} at high school.\vspace{1mm}
%\item Ranked \textbf{10\textsuperscript{th}} in \textbf{National Astronomy Competition} among more than 1000 participants


%----------------------------------------------------------------------------------------
%	PUBLICATION SECTION
%----------------------------------------------------------------------------------------

%----------------------------------------------------------------------------------------
%	M.S PROJECTS SECTION
%----------------------------------------------------------------------------------------
%\section{M.Sc Thesis}
%\begin{itemize}
	%\item Designing an electricity market among Microgrids is considered.
	%Two different types of \textbf{Electricity market} has been proposed in this thesis so as to define \textbf{day-ahead} and \textbf{real-time price} of a market. A two-stage optimization method has been proposed to maximize Multi-Microgrid profit. Moreover, the operation of \textbf{CHP} (combined heat and power) has been taken into account. the uncertainty is also considered using \textbf{stochastic} methods.  
	
	%\begin{itemize}
		%\item Active Filter Simulation.
		%\item\textcolor{gray}{Supervisor: \textbf{Dr. Jadid} } 
	%\end{itemize}
%\end{itemize}

%----------------------------------------------------------------------------------------
%	B.Sc PROJECTS SECTION
%----------------------------------------------------------------------------------------
%\section{B.Sc Thesis}
%\begin{itemize}
%\item Segmenting brain MRI images using Dempster-Shafer rule of Combination and Fuzzy logic.\\
%Dempster-Shafer method have been used to combine results of different images from the dataset to obtain fuzzy rules. Results have been tested on BrainWeb and IBSR datasets. On the next stage, Fuzzy type \rom{2} approach was applied to images to make better results. This project is divided into the following sections:

%\begin{itemize}
%\item Extracting Fuzzy Rules using Dempster-Shafer.
%\item Simulating in Matlab.
%\item Comparing the results with other methods.
%\item Active Filter Simulation.
%\item\textcolor{gray}{20/20,Supervisor: \textbf{Dr. Sharifian} \hfill{2015}} 
%\end{itemize}
%\end{itemize}



%----------------------------------------------------------------------------------------
%	INTERNSHIP SECTION
%----------------------------------------------------------------------------------------
%\section{Internship}
%\begin{itemize}
%\item Segmenting MRI images using \textbf{GMM} and prior probabilities.\\
%First a skull stripping algorithm had been developed, then GMM was applied to images. Next, \textbf{SPM8} (an open source toolbox for MRI segmentation and analysis in Matlab) was studied and some changes were applied to it to make it usable for 2D images.
%\begin{itemize} 
%\item\textcolor{gray}{19/20, Supervisor: \textbf{Dr. Nasiraei Moghaddam} \hfill{Medical Images Lab 2014}} 
%\end{itemize}
%\end{itemize}

%----------------------------------------------------------------------------------------
%	BEST COURSES SECTION
%----------------------------------------------------------------------------------------



%-----------------------------------------------------------------------------------------
%	TEACHING EXPERIENCE SECTION
%-----------------------------------------------------------------------------------------
\section{Work Experiences}
Generally, my research experiences in the university were about the orbit and attitude dynamics of the spacecraft in the two or three-body regimes. After I graduated, I tried to use what I learned in real industrial applications. So, since Jan. 2020, I have worked as an R\&D expert in a startup company ("\textbf{LEOCT}"). In LEOCT, I have been faced with various tasks such as \textbf{Finding Periodic Solutions in Complex Environments}, \textbf{Ephemeris Calculations}, \textbf{Precise Orbit Determinations}, and \textbf{Solar Sailing}. All of the mentioned tasks were implemented in \textbf{MATLAB} and \textbf{Octave} programming languages.


\section{Research Experiences}

\begin{itemize}
\item{Professional in Differential Correction Algorithms (Shooting Methods, as a mean for generating periodic orbits (or attitude motions) in multi-body systems)}
\item{Professional in using and handling search methods for finding periodic attitude or orbit behaviors (Poincar\'e Sections, etc.)}
\item{Investigation on Machine Learning and Optimization methods for Astrodynamics Applications}
\item{Investigation on Ephemeris calculation process (providing precise orbit and clock products for LEO constellations), and other GNSS aspects (orbital mechanics related)}
\end{itemize}

%----------------------------------------------------------------------------------------
%	PROFESSIONAL EXPERIENCE SECTION
%----------------------------------------------------------------------------------------

%----------------------------------------------------------------------------------------
%	LANGUAGE SKILLS SECTION
%----------------------------------------------------------------------------------------
%\section{Language Skills}

%\begin{itemize}
%	\item \textbf{English} \hspace{6 pt} Fluent
%	\\\textbf{TOEFL} : Toefl Scores will be added soon

%	\item \textbf{Persian} \hspace{5 pt} Native
%\end{itemize} 


%----------------------------------------------------------------------------------------
%	COMPUTER SKILLS SECTION
%----------------------------------------------------------------------------------------
\section{Skills}

\textbf{English} : Fluent (TOEFL score: 96, R:28, L:28, S:20, W:20) \\
\textbf{Persian} : Native
\\
\\
\textbf{Software}
\begin{itemize}
\item{STK}
\item{SOLIDWORKS}
\item{CATIA}
\end{itemize}
\textbf{programming languages}
\begin{itemize}
\item {Octave}
\item {Matlab}
\item{Python}
\end{itemize}
\textbf{General Softwares}
\begin{itemize}
\item {LaTeX}
\item {Microsoft Office}
\end{itemize}
%----------------------------------------------------------------------------------------
%	Hardware Section SECTION
%----------------------------------------------------------------------------------------

%----------------------------------------------------------------------------------------
%	HONOR SECTION
%----------------------------------------------------------------------------------------
%----------------------------------------------------------------------------------------
%	ACADEMIC PROJECTS SECTION
%----------------------------------------------------------------------------------------
\section{Academic Projects}

\begin{itemize}

\item Analysis of the Spacecraft Attitude Dynamics in the CR3BP by the Mean of Maximum Gravity Torque Surfaces.
\begin{itemize}
	\item \textcolor{gray}{Supervisor: \textbf{Dr. Majid Bakhtiari } }
\end{itemize}

\item Design, Implementation and Verification of the Attitude Determination and Control Algorithms for the DelFFi Satellites.
\begin{itemize}
	\item \textcolor{gray}{Supervisor: \textbf{ Dr. Seyed Majid Esmaeilzadeh} }
\end{itemize}

\item Investigating the Periodic Solutions of the Coupled Orbit-Attitude Perturbed Circular Restricted Three-Body Problem. 
\begin{itemize}
	\item \textcolor{gray}{Supervisor: \textbf{Dr. Majid Bakhtiari} }
\end{itemize}


\item Simulation of MEMS Inertial Earth Sensor Dynamic for Measuring Gravity Gradient Torque in Low Earth Orbit.
\begin{itemize}
	\item \textcolor{gray}{Supervisor: \textbf{Dr. Meisam farajollahi} }
\end{itemize}


\item Investigating the Effect of Eccentricity and Mass Ratio of Primaries on the Structure of Lyapunov Orbits.
\begin{itemize}
	\item \textcolor{gray}{Supervisor: \textbf{Dr. Kamran Daneshjoo, Dr. Majid Bakhtiari}}
\end{itemize}

\item Satellite Lifetime Simulation.
\begin{itemize}
	\item \textcolor{gray}{Supervisor: \textbf{Bahman Ghorbani Vaghei} }
\end{itemize}

\end{itemize}

%----------------------------------------------------------------------------------------
%	HOBBIES AND INTERESTS SECTION
%----------------------------------------------------------------------------------------
%\section{Hobbies}
%\begin{itemize}
%	\item Reading books, Swimming, Movies, Trying to understand how stuff works 
%\end{itemize}

%----------------------------------------------------------------------------------------
%	References and Proofs
%----------------------------------------------------------------------------------------
\section{References}
\begin{itemize}
	\item \textbf{Dr. Majid Bakhtiari}\\
	School of New Technologies, Iran University of Science and Technology, Tehran, Iran\\
	Email: bakhtiari\_m@iust.ac.ir\\
	Tel: +98-912-320-6574 ~~~~
	\textbf{\href{https://scholar.google.com/citations?user=Ha4mcYoAAAAJ&hl=en} {\underline{Google scholar}}}
	
	
	\item \textbf{Dr. Kamran Daneshjoo}\\
	Department of Mechanical engineering, Iran University of Science and Technology, Tehran, Iran\\
	Email: kjoo@iust.ac.ir\\
	Tel: +98-21-77240570 ~~~~
    \textbf{\href{http://www.iust.ac.ir/content/873/Daneshjo} {\underline{Home page}}}
	
%	\item \textbf{Dr. Bahman Ghorbani Vaghei}\\
%	School of Railway Engineering, Iran University of Science and Technology, Tehran, Iran\\
%	Email: bahman\_gh@iust.ac.ir \\
%	Tel: +98-21-77491029\\
 %   \textbf{\href{https://scholar.google.com/citations?user=i5gcBVQAAAAJ&hl=en&oi=sra} {\underline{Google scholar}}}
    
    \item \textbf{Dr. Meisam farajollahi}\\
	School of New Technologies, Iran University of Science and Technology, Tehran, Iran\\
	Email: farajollahi@iust.ac.ir\\
	Tel: +98-21-73225825~~~~
	\textbf{\href{https://scholar.google.com/citations?hl=en&user=wFljLq8AAAAJ&view_op=list_works&sortby=pubdate} {\underline{Google scholar}}}


\end{itemize}

\end{document}

%----------------------------------------------------------------------------------------
%----------------------------------------------------------------------------------------